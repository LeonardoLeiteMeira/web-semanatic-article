\chapter{Tools}

This section is used to describe the tools used in the development of the project, going through the structuring of the database and the construction of the API.

\section{Protégé}

Protégé \cite{protege} is a free, open source ontology editor and a knowledge management system, was developed by the Stanford Center for Biomedical Informatics Research at the Stanford University School of Medicine. The software is the most popular and widely used ontology editor in the world, works in Windows, Linux, and MacOS.

This tool allows mapping different users of different social media and their connection, and and build queries to search the database. The details of entities and relationships are described in chapter 3.

\section{Python}
Python \cite{python} is a general purpose programming language that in the scope of work was used to automate processes and manipulate databse, abstracting the necessary steps and also applying the necessary validations and transformations before performing the insertions. 
This language is also used to provide the information stored in the database through an API (Application Programming Interface), which makes it possible to create a graphical interface for user interactions. The framework used to build the API was FastAPI \cite{fastapi}.

