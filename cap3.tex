\chapter{Conceptual solution}
%The first necessarily step to the development of the work is make the abstractions, for that we classify and separate the entities, relationships and attributes of work. The data base is composed by a couple of entities called by “Social Media”, "Person" and Connection, a couple types of relationship, the "HAS" to connect one person to a social network, and another relationship to represent connections type between the social networks, as the possible following labels: "PERSONAL", "INFLUENCE", "PROFESSIONAL\_RELATIONSHIP", e "CONTENTS".

%Such abstractions that were citated anteriorly are codified in a program that automatize the insertion process and population of Neo4J, applying some validations and a defined logic to several connections as the verification of a relationship type, and can be find at the following \href{https://github.com/LeonardoLeiteMeira/process_CSV_to_graph_database}{repositório}. To use is necessary substitute connections data of Neo4J to the local machine.



\section{Entities}
System data is based on social medias entities (SocialMedia), users (User), and the connection entity (Connection), each with its own properties.

\subsection{Social Medias}  
    
The Social Media is responsible for represent one social network. A social network can be connected with a user and connection. By this way, the social networks that were chosen to be part of the project initially are currently the main utilized, being them the Instagram, Facebook, LinkedIn, Ticktok, Youtube, Pinterest e Twitter.

The attributes of each social network are:


\begin{itemize}

\item Id: Single identifier

\item Name: Social network name that the entity refers to 

\item Description: Text that represent a description of social media

\end{itemize}


\subsection{User}

The “User” entity represent the user that utilize determinated social networks. One person can be related to many others different social networks, and yours relations with another people happen through connection entity. 

One instance of the “Person” entity have the following attributes:


\begin{itemize}
\item Identifier: A single identifier to that user
\item Name: User’s first name
\item Surname: User’s last name
\item Email
\item Initial date: Date the user signup into the system
\item Updated date: Update date of the information in the system
\end{itemize}

\subsection{Connection}
The connection entity represents the connection between two users of a social network. A connection is linked to two different users, representing the relationship between them, and also to a social media, storing where the relationship between those users takes place. The connection entity has the following properties:

\begin{itemize}
    \item Identifier: A single identifier to that connection
    \item Initial date: Date the connection was registered in the system
    \item Final date: Date the connection was broken
    \item Relationship type: Represents the type of relationship between those two users
\end{itemize}


The relationship type takes into account the type of connection and the social networks used by two users, and can be of the following types:

\begin{itemize}
    \item Personal: Follow each other on networks for personal use
    \item Professional: Follow each other in professional networks
    \item Content consumption: Follow each other on networks focused on video content
    \item Influence: Represents the relationship of influence, when a person does not follow another but is followed by him. This type is used for influence score.
\end{itemize}


\section{Relationships}

The relationships between the entities describe the behavior that are being charted by the graphs. Within the proposed system there are three types of relationship that are described as follows:

\begin{itemize}
    \item Relationship between a user and the social media: Vertex with the properties username, date joined, status, date deleted (optional).
    \item Relationship between user and connection: Used to store connection information between two users in a social network.
    \item Relationship between connection and social media: Used to represent in which social media the relationship between the two users exists
\end{itemize}


\section{Queries}

Given the database structure presented and the objectives of this work, some queries were created to return the desired information. 
The querie below returns the social media of a given person.
\begin{verbatim}
    SELECT ?socialName
    WHERE {
     ?subject rdf:type :User;
              :email ?name ;
              :id ?id ;
              :IsOwnerOf ?social .
      ?social :name ?socialName
      FILTER (?id = 2)
    }
\end{verbatim}


The next query is intended to return all people connected to someone else's social networks.

\begin{verbatim}
    SELECT DISTINCT ?email ?id ?idConnection ?connectionEmail
    WHERE {
     ?subject rdf:type :User;
              :email ?email ;
              :id ?id ;
              :HasRelationship ?relations .
      ?relations :With ?connection .
      ?connection :id ?idConnection ;
                  :email ?connectionEmail
      FILTER (?id = 1)
    }
\end{verbatim}

The following query looks for the first-level connections for a relationship type.

\begin{verbatim}
    SELECT DISTINCT ?recommends
    WHERE {
     ?subject rdf:type my:Person;
              my:PersonID ?id ;
              my:HAS	?social;
             OPTIONAL {
                ?type rdfs:subPropertyOf* my:RELATIONSHIP .
                ?social ?type ?conections .
                ?social	?type ?recommends .
                FILTER NOT EXISTS {?recommends ?type ?social}
              } 
      FILTER ( ?id = 1 && !isBlank(?recommends) )s
    }
\end{verbatim}

% The following query performs an in-depth search using the plugion APOC \cite{apoc} available in the Neo4J system. In the first line, the social network with “Name” defined as “Instagram” of the person “Alice“ is selected. In the second line, the search for node S (Alice's Instagram) begins, then the search for vertices is performed, with a maximum depth of 5. The result of this query are the suggestions of possible new connections of a personal nature that can be made .

% \begin{verbatim}
% match (p:Person{Name:"Alice"})-[]-(s:SocialMedia{Name:"Instagram"})
% call apoc.path.spanningTree(s,{relationshipFilter: "PERSONAL",minLevel:0,maxLevel: 5})
% YIELD path
% RETURN path, p;
% \end{verbatim}

% For other types of recommendations just replace the "relationshipFilter" property to "INFLUENCE", to have influence recommendations, to "PROFESSIONAL\_RELATIONSHIP" to have professional connection recommendations, or "CONTENTS" to have content recommendations.


The following query is intended to return the influence score that a given person has.

\begin{verbatim}
    SELECT (count ( distinct ?relations ) AS ?level)
    WHERE {
       ?relations rdf:type :Influence .
       ?relations :With ?user .
       ?user :id ?id .
       FILTER (?id = 4)
    }GROUP BY ?id
\end{verbatim}


% This last query aims to identify someone's level of influence within a network. In the first line, the social networks of a person are selected to be evaluated for influence. In the second line, all the social networks of a network that must be evaluated, which are influenced are selected, and finally it is defined which network the level of influence is being evaluated, in this case a social network with the property "Username" with the value " LauraInstagram”.
% \begin{verbatim}
% match(person:Person {Name: "Leonardo"})-[HAS]->(socialMedia:SocialMedia)
% match(socialMedia)<-[f:INFLUENCE]-(socialMediaInOtherSubGraph:SocialMedia)
% match(socialMediaInOtherSubGraph:SocialMedia)-
%     []-(subGraph:SocialMedia{Username:"LauraInstagram"})

% return count(f)
% \end{verbatim}

\section{API}
To use the system \cite{MetaSocialMediaAPI}, it is necessary to create an access, and log in to the platform. In this way, it is possible to have access to endpoints that return recommendations and level of influence. Existing endpoints are:
\begin{itemize}
    \item List Social Medias of a user: This endpoint receives in the path parameter the id of the user that must be listed the social medias. Once the user exists in the database, a list of social media names referring to the informed user will be returned.
    \item A user's connections: This endpoint returns a list of people who are connected with a given user. The id of a user to be searched for is a mandatory parameter (path parameter), and as optional parameters (query parameter) we have the type of connection, and a certain social media. If any mandatory parameter is not informed, the response will contain all possibilities.
    \item Connection suggestions: This endpoint returns a list of possible connections for a given user whose id is informed as a parameter (path parameter). As optional parameters (query parameter) there are the type of connection, and a certain social media, to suggest connections only from a certain context. The return from this endpoint is basically a search in the graph of connections that the user's connections make (small in-depth search).
    \item Level of influence: This endpoint returns the level of influence of a given user whose id is passed as a parameter (path parameter). As optional parameters (query parameter) there are a specific social media, and a sub-graph that would be someone else's social network (user id). This calculation is based on the number of influence vertices that arrive in a user's social media, which can be specifically from a social media, or from another user's social network, making it possible to analyze someone's level of influence within another context.
    \item Create access: Endpoint to create access to API. It receives in the body of the request in JSON (Java Script Object Notation) format, email, name and password.
    \item Login: Receive formdata with username (email) and password, and them return a JWT token with 24 hours validity
\end{itemize}





